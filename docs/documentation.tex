\documentclass[12pt]{article}
\usepackage{float}
\floatstyle{plain}
\newfloat{program}{thp}{lst}
\floatname{program}{Listing}
\usepackage[hidelinks]{hyperref}
\usepackage{xcolor}
\usepackage{listings}
\lstset{
    language=C,
    tabsize=4,
    basicstyle=\scriptsize\ttfamily,
    breaklines=true,
    frame=l,
    rulecolor=\color{black},
    numbers = left,
    numberstyle=\tiny\ttfamily,
    numbersep=10pt,
    stepnumber = 1,
    keywordstyle=\bf\color{blue},
    commentstyle=\color{black!30},
    showspaces=false,
    showtabs=false,
    showstringspaces=false,
    stringstyle=\color{brown!50!orange},
    morekeywords = {
        asm,
        uint8,
        uint16,
        uint32,
        int8,
        int16,
        int32
    }
}

\newcommand{\code}[3]{\lstinputlisting[firstline=#1, lastline=#2, firstnumber=#1]{../#3}}

\parindent=0em
\parskip=.8em

\title{\huge\bf\vspace{-1em} Alarm Clock \\ Embedded Systems Project Report\\\vspace{2em}{\large\normalfont Monday 10:00 lab} }

\author{
    Jakub Pawlak \\
    \texttt{234767@edu.p.lodz.pl}
    \and
    Artur Pietrzak \\
    \texttt{234768@edu.p.lodz.pl}
    \and
    Juliusz Szymajda \\
    \texttt{234769@edu.p.lodz.pl}
}
\begin{document}
    \maketitle
    \clearpage
    \large
    \section*{Devices used:}
    Eduboard LPC2148 v1.0
    \section*{Interfaces used:}
    GPIO, I\textsuperscript{2}C, SPI
    \section*{Devices used:}
    \begin{enumerate}
        \item LCD display
        \item RTC
        \item Button
        \item Joystick
        \item Buzzer
        \item Timer
        \item EEPROM
    \end{enumerate}
    \clearpage
    \tableofcontents
    \listof{program}{Code Listings}
    \clearpage

    \section{Project Description}
    \subsection{General description}
    \section{Peripherals and interface configuration}
    \subsection{LCD Display}
    \begin{program}[h]
        \code{62}{95}{display.h}
        \caption{LCD setup function}
    \end{program}

    \clearpage
    \section{Failure Mode and Effect Analysis}
    \begin{table}[H]\centering
        \newcommand{\critical}{\color{red} Critical}
        \begin{tabular}{|l|c|}
            \hline
            \bf Component   &   \bf Severity    \\\hline
            Microcontroller &   \critical       \\\hline
            Power Supply    &   \critical\footnotemark       \\\hline
            RTC             &   \critical       \\\hline
            LCD Display     &   High            \\\hline
            Speaker         &   High            \\\hline
            Button          &   High            \\\hline
            Joystick        &   High            \\\hline
        \end{tabular}
        \caption{Severity of component's failure}
    \end{table}
    \footnotetext[1]{Long-term power supply failures are of critical severity, but in case of short pause in power delivery, the system is able to recover using the RTC and the data stored in EEPROM}
\end{document}
