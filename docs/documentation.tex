\documentclass[11pt]{article}
\usepackage{float}
\floatstyle{plain}
\newfloat{program}{thp}{lst}
\floatname{program}{Listing}
\usepackage[hidelinks]{hyperref}
\usepackage{xcolor}
\usepackage{listings}
\lstset{
    language=C,
    tabsize=4,
    basicstyle=\scriptsize\ttfamily,
    breaklines=true,
    frame=l,
    rulecolor=\color{black},
    numbers = left,
    numberstyle=\tiny\ttfamily,
    numbersep=10pt,
    stepnumber = 1,
    keywordstyle=\bf\color{blue},
    commentstyle=\color{black!30},
    showspaces=false,
    showtabs=false,
    showstringspaces=false,
    stringstyle=\color{brown!50!orange},
    morekeywords = {
        asm,
        uint8,
        uint16,
        uint32,
        int8,
        int16,
        int32,
	  RTC_Time
    }
}

\newcommand{\code}[3]{\lstinputlisting[firstline=#1, lastline=#2, firstnumber=#1]{../#3}}

\parindent=0em
\parskip=.8em

\title{\huge\bf\vspace{-1em} Alarm Clock \\ Embedded Systems Project Report\\\vspace{2em}{\large\normalfont Monday 10:00 lab} }

\author{
    Jakub Pawlak \\
    \texttt{234767@edu.p.lodz.pl}
    \and
    Artur Pietrzak \\
    \texttt{234768@edu.p.lodz.pl}
    \and
    Juliusz Szymajda \\
    \texttt{234769@edu.p.lodz.pl}
}
\begin{document}
    \maketitle
    \clearpage
    \large
    \section*{Devices used:}
    Eduboard LPC2148 v1.0
    \section*{Interfaces used:}
    GPIO, I\textsuperscript{2}C, SPI
    \section*{Devices used:}
    \begin{enumerate}
        \item LCD display
        \item RTC
        \item Button
        \item Joystick
        \item Buzzer
        \item Timer
        \item EEPROM
    \end{enumerate}
    \clearpage
    \tableofcontents
    \listof{program}{Code Listings}
    \clearpage

    \section{Project Description}
    \subsection{General description}
    The project is an digital clock with the ability to set an alarm for a given hour.
    The device displays current time on the LCD display.
    Using button and joystick it is possible to set the current time, or set an alarm.
    If the alarm is set and the alarm time is the current time, the buzzer emits a sound until the alarm is turned off.

    \subsection{Setting the alarm}
    The alarm setting mode it entered by pressing down the joystick center push-down switch.
    When in alarm setting mode, the user can change the alarm time by selecting the digit with left/right movement, and selecting the value with up/down movement.
    After the correct time is selected, alarm is set by pressing the button.

    When alarm is turned on, it can be turned off by short press of the button.

    \subsection{Changing the current time}
    The time setting mode is entered by holding down the button.
    Then, the time is set with the joystick in the similar way, the alarm is set.
    After setting the time it is set by short press of the button.

    \subsection{Turning off the alarm}
    When the alarm clock starts emmiting sound, the user can turn it off by pressing the button.

    \section{Peripherals and interface configuration}
    \subsection{LCD Display}
    \begin{program}[h]
        \code{63}{96}{display.h}
        \caption{LCD setup function}
    \end{program}

    \subsection{GPIO}

    \subsection{I\textsuperscript{2}C}

    \subsection{SPI}

    \subsection{Debounce Timer}
    
    \subsection{Joystick}
    	\begin{lstlisting}[caption = {Joystick get input function}]
char getJoyInput(void) {
    if (((IOPIN0 & KEYPIN_UP) == 0)) {
		return 'u';
	}
	if (((IOPIN0 & KEYPIN_DOWN) == 0)) {
		return 'd';
	}
	if (((IOPIN0 & KEYPIN_RIGHT) == 0)) {
		return 'r';
	}
	if (((IOPIN0 & KEYPIN_LEFT) == 0)) {
		return 'l';
	}
	if (((IOPIN0 & KEYPIN_CENTER) == 0)) {
		return 'c';
	}
    return 0;
}
	\end{lstlisting}
	The function checks if the joystick is utilized and return char corresponding to the current position of the joystick.

	\begin{lstlisting}[caption = {Await joystick input function}]
char waitForJoyInput(void){
	while (1) {
		char result = getJoyInput();
		if (result != 0)
			return result;
		delay37us();
	}
}
	\end{lstlisting}
This function runs in an infinite loop, which ends when the result of the JoyInputFunction is not 0 (when there is some input from the joystick).

	\begin{lstlisting}[caption = {Get button input function}]
uint8 isButtonPressed(void) {
	if ((IOPIN0 & PIN_BUTTON) == 0) {
		return TRUE;
	}
	else {
		return FALSE;
	}
}
	\end{lstlisting}
This function check wether the button on the board is pressed.

    \subsection{RTC}
	\begin{lstlisting}[caption = {RTC initialization}]
	RTC_Time setTime, alarmTime, currentTime;


    CCR = 0x01;
	\end{lstlisting}

	First, the structs for time to set, alarm time and current time are created. The RTC interrupts are disabled by writing 0x0 to Interrupt Location Register (ILR).  The clocks tick counter is also reset by writing 0x02 to the Clock Control Register (CCR), and disabled by writing 0x0. Next, the Counter Increment Interrupt (CIIR), which is responsible for incrementing seconds, is disabled. 

Next, the prescaler is configured. Prescaler is a circuit used to reduce a high frequency electrical signal (for example from RTC) to a lower frequency by integer division. After calculating the value of a prescaler integer:
\[ (PCLK/32768) – 1 \]
and value of prescaler fraction:
\[ PCLK – ((PREINT + 1) * 32768) \]
Those values are written to Prescaler Integer Register (PREINT) and Prescaler Fraction Register (PREFRAC) registers.

Finally, the time is read from EEPROM, written into setTime struct and set using RTC\_SetTime function. At the end of initialization 0x01 is writtent to Clock Control Register (CCR), which enables the timer counters.

	\begin{lstlisting}[caption = {RTC struct}]
typedef struct
{
	uint8 seconds;
	uint8 minutes;
	uint8 hours;
} 
	\end{lstlisting}
	The struct consists of three 8 bit unsigned integer numbers, each one for seconds, minutes and hours fields.

	\begin{lstlisting}[caption = {Setting time in the register function}]
void RTC_SetTime(RTC_Time Time)
{
	SEC = Time.seconds;
	MIN = Time.minutes;
	HOUR = Time.hours;
}
	\end{lstlisting}
	The function uses data stored in the parsed struct and writes it to counters from Time Register Group. SEC and MIN are 6 bit, and the HOUR counter is 5 bit.

	\begin{lstlisting}[caption = {Setting alarm time in the register function}]
void RTC_Set_AlarmTime(RTC_Time AlarmTime)
{
	ALSEC = AlarmTime.seconds;
	ALMIN = AlarmTime.minutes;
	ALHOUR = AlarmTime.hours;
}
	\end{lstlisting}
	The function uses data stored in the parsed struct and writes it to counters from Alarm Register Group. Similarly to the Time Register Group, ALSEC and ALMIN are 6 bit, and the ALHOUR counter is 5 bit.

	\begin{lstlisting}[caption = {Getting time from the register function}]
RTC_Time RTC_GetTime(void)
{
	RTC_Time time;

	time.seconds = SEC;
	time.minutes = MIN;
	time.hours = HOUR;

	return time;
}
	\end{lstlisting}
	The function takes data from SEC, MIN and HOUR registers stored in Time Register Group, assigns it to the newly created RTC\_Time struct and returns it.

	\begin{lstlisting}[caption = {Getting alarm time from the register function}]
RTC_Time RTC_GetAlarmTime(void)
{
	RTC_Time AlarmTime;

	AlarmTime.seconds = ALSEC;
	AlarmTime.minutes = ALMIN;
	AlarmTime.hours = ALHOUR;

	return AlarmTime;
}
	\end{lstlisting}
	Analogously, to the RTC\_GetTime function, this function takes data from ALSEC, ALMIN and ALHOUR registers stored in Alarm Register Group, assigns it to the newly created RTC\_Time struct and returns it.

	\begin{lstlisting}[caption = {RTC\_Time struct comparator function}]
bool RTC_Compare(RTC_Time t1, RTC_Time t2)
{
	return (t1.seconds == t2.seconds
		&& t1.minutes == t2.minutes
		&& t1.hours == t2.hours);
}
	\end{lstlisting}
	The function is used to compare two RTC\_Time structs by comparing each field of the struct (seconds, minutes and hours).

    \section{Failure Mode and Effect Analysis}
    \begin{table}[H]\centering
        \newcommand{\critical}{\color{red} Critical}
        \begin{tabular}{|l|c|}
            \hline
            \bf Component   &   \bf Severity                \\\hline
            Microcontroller &   \critical                   \\\hline
            Power Supply    &   \critical\footnotemark      \\\hline
            RTC             &   \critical                   \\\hline
            LCD Display     &   High                        \\\hline
            Speaker         &   High                        \\\hline
            Button          &   High                        \\\hline
            Joystick        &   High                        \\\hline
        \end{tabular}
        \caption{Severity of component's failure}
    \end{table}
    \footnotetext[1]{Long-term power supply failures are of critical severity, but in case of short pause in power delivery, the system is able to recover using the RTC and the data stored in EEPROM}

    \addcontentsline{toc}{section}{References}
    \nocite{*}
    \bibliographystyle{unsrt}
    \bibliography{references}
\end{document}
