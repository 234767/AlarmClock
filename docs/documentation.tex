\documentclass[10pt]{article}
\usepackage{float}
\floatstyle{plain}
\newfloat{program}{thp}{lst}
\floatname{program}{Listing}
\usepackage[hidelinks]{hyperref}
\usepackage{xcolor}
\usepackage{listings}
\lstset{
    language=C,
    tabsize=4,
    basicstyle=\scriptsize\ttfamily,
    breaklines=true,
    frame=l,
    rulecolor=\color{black},
    numbers = left,
    numberstyle=\tiny\ttfamily,
    numbersep=10pt,
    stepnumber = 1,
    keywordstyle=\bf\color{blue},
    commentstyle=\color{black!30},
    showspaces=false,
    showtabs=false,
    showstringspaces=false,
    stringstyle=\color{brown!50!orange},
    morekeywords = {
        asm,
        uint8,
        uint16,
        uint32,
        int8,
        int16,
        int32,
	  RTC_Time
    }
}

\newcommand{\code}[3]{\lstinputlisting[firstline=#1, lastline=#2]{../#3}}

\parindent=0em
\parskip=.8em

\title{\huge\bf\vspace{-1em} Alarm Clock \\ Embedded Systems Project Report\\\vspace{2em}{\large\normalfont Monday 10:00 lab} }

\author{
    Jakub Pawlak \\
    \texttt{234767@edu.p.lodz.pl}
    \and
    Artur Pietrzak \\
    \texttt{234768@edu.p.lodz.pl}
    \and
    Juliusz Szymajda \\
    \texttt{234769@edu.p.lodz.pl}
}
\begin{document}
    \maketitle
    \clearpage
    \large
    \section*{Devices used:}
    Eduboard LPC2148 v1.0 \\[1em]
    No external devices used
    \section*{Interfaces used:}
    GPIO, I\textsuperscript{2}C, SPI
    \section*{Devices used:}
    \begin{enumerate}
        \item LCD display
        \item RTC
        \item Button
        \item Joystick
        \item Buzzer
        \item Timer
        \item EEPROM
    \end{enumerate}
    \clearpage
    \tableofcontents
    \listof{program}{Code Listings}
    \clearpage

    \section{Project Description}
    \subsection{General description}
    The project is an digital clock with the ability to set an alarm for a given hour.
    The device displays current time on the LCD display.
    Using button and joystick it is possible to set the current time, or set an alarm.
    If the alarm is set and the alarm time is the current time, the buzzer emits a sound until the alarm is turned off.

    \subsection{Project functionalities}
    \begin{table}[H]\centering
        \begin{tabular}{|l|l|l|}
            \hline
            \bf Functionality   & \bf Person Responsible    & \bf Implementation status                     \\ \hline
            Timer               & Jakub Pawlak              & Implemented                                   \\ \hline
            RTC                 & Artur Pietrzak            & \color{orange!80!red} Partially implemented   \\ \hline
            SPI                 & Artur Pietrzak            & \color{red} Not implemented                   \\ \hline
            LCD Display         & Jakub Pawlak              & Implemented                                   \\ \hline
            Buttons \& Joystick & Artur Pietrzak            & Implemented                                   \\ \hline
            Buzzer              & Jakub Pawlak              & \color{red} Not implemented                   \\ \hline
            EEPROM              & Juliusz Szymajda          & \color{red} Not implemented                   \\ \hline
            I2C                 & Juliusz Szymajda          & \color{red} Not implemented                   \\ \hline
        \end{tabular}
        \caption{Project functionalities and responsible persons}
    \end{table}

    \subsection{Setting the alarm}
    The alarm setting mode it entered by pressing down the joystick center push-down switch.
    When in alarm setting mode, the user can change the alarm time by selecting the digit with left/right movement, and selecting the value with up/down movement.
    After the correct time is selected, alarm is set by pressing the button.

    When alarm is turned on, it can be turned off by short press of the button.

    \subsection{Changing the current time}
    The time setting mode is entered by holding down the button.
    Then, the time is set with the joystick in the similar way, the alarm is set.
    After setting the time it is set by short press of the button.

    \subsection{Turning off the alarm}
    When the alarm clock starts emmiting sound, the user can turn it off by pressing the button.

    \section{Peripherals and interface configuration}
    
    \subsection{GPIO}
    
    \subsection{LCD Display}
    \subsubsection{Setup}
    We begin setting up the lcd by configuring the IODIR registers to set the appropriate pins as output and clear them.
    The pins used are data pins P1.16 to P1.23, represented as {LCD\_DATA}, lcd enable pin P1.25 (LCD\_E), lcd \mbox{read/write} pin P0.22 (LCD\_RW), register select P1.24 (LCD\_RS), and backlight P0.30 (LCD\_BACKLIGHT).

    Next, we proceed by sending a command $(38)_{16}$, which represents the `Function Set' instruction.
    We set the data length to 8 bits, and the number of display lines to 2.

    Next, we send display on/off command, to reset the display.

    Then, we clear the display and send the $(06)_{16}$ command, which sets the cursor direction to `increment'.

    Finally, we turn on the display, without cursor, and set cursor to home position.

    All of the values to send, were taken from the instruction table in the display driver manual \cite[p.24]{display-man}

    \begin{program}[H]
        \code{63}{96}{display.h}
        \caption{LCD setup function}
    \end{program}

    \subsubsection{Sending data to the display}

    \begin{program}[H]
        \code{16}{52}{display.h}
        \caption{Functions for using the display}
        \label{lst:display-util-functions}
    \end{program}

    The core of interaction with the LCD is the \texttt{writeLCD} function, which takes 2 parameters - register (0 --- instruction or 1 --- data),
    and the actual 8-bit data we want to send. 
    We begin by setting the register select pin (lines 6--9).

    Before we send any data, we clear the read/write pin, to indicate that we want to write data to the display.
    Then we clear the data pins, and set the value in line 13.
    We first have to shift the value by 16 bits to the left, because we use pins P1.16--P1.23, and then we mask them with LCD\_DATA to make sure that we do not set any other pins.

    \pagebreak
    After setting the data pins, we set the LCD\_E pin, which starts the data read. 
    After some delay, we stop the data transmission by clearing LCD\_E pin and wait an additional delay before the next piece of data can be sent.

    \subsubsection{Moving the cursor}
    The text that is currently displayed on the LCD is stored in so-called `display data RAM' (DDRAM). 
    The cursor is located at whatever address is currently stored at the DDRAM address counter \cite[p.21]{display-man}.

    Therefore, we can change the cursor position, by changing the address in the AC.
    Luckily, the LDC controller has an instruction for just that purpose.
    It is used by sending a command with bit 7 set, followed by 7-bit DDRAM address.

    The first row is addresses $(00)_{16}$ -- $(27)_{16}$, and the second row is addresses $(40)_{16}$ -- $(67)_{16}$ \cite[p.11]{display-man}.

    Therefore, for setting the cursor position, we created the following function:
    \begin{program}[H]
        \code{108}{116}{display.h}
        \caption{Function for controlling the cursor position}
    \end{program}

    Firstly, we limit the column to not be greater than 39, as the DDRAM only has 40 characters per line.

    Then, we send the `Set DDRAM address' command with the appropriate address.
    If we want to set the position in the first line, we send $(80)_{16}$ (7-th bit set, indicating command) followed by the column, as the 1st line address starts from 0.
    If we want to set the position in the second line, instead of $(80)_{16}$ we use $(\mathrm{C0})_{16}$, which is $(80)_{16}$ (7th bit) + $(40)_{16}$ (starting address of the 2nd line).

    \subsubsection{Writing text}

    Writing text to display is very simple, we just send data, and the cursor automatically increments position after each character.
    We just loop over the array of characters and write each one using the \texttt{LcdData} function from listing \ref{lst:display-util-functions}.

    \begin{program}[H]
        \code{118}{124}{display.h}
        \caption{Function for writing text to the display}
    \end{program}

    \subsection{I\textsuperscript{2}C}

    \subsection{SPI}

    \subsection{Debounce Timer}
    
    \subsection{Joystick}
    	\begin{lstlisting}[caption = {Joystick get input function}]
char getJoyInput(void) {
    if (((IOPIN0 & KEYPIN_UP) == 0)) {
		return 'u';
	}
	if (((IOPIN0 & KEYPIN_DOWN) == 0)) {
		return 'd';
	}
	if (((IOPIN0 & KEYPIN_RIGHT) == 0)) {
		return 'r';
	}
	if (((IOPIN0 & KEYPIN_LEFT) == 0)) {
		return 'l';
	}
	if (((IOPIN0 & KEYPIN_CENTER) == 0)) {
		return 'c';
	}
    return 0;
}
	\end{lstlisting}
	The function checks if the joystick is utilized and return char corresponding to the current position of the joystick.

	\begin{lstlisting}[caption = {Await joystick input function}]
char waitForJoyInput(void){
	while (1) {
		char result = getJoyInput();
		if (result != 0)
			return result;
		delay37us();
	}
}
	\end{lstlisting}
This function runs in an infinite loop, which ends when the result of the JoyInputFunction is not 0 (when there is some input from the joystick).

	\begin{lstlisting}[caption = {Get button input function}]
uint8 isButtonPressed(void) {
	if ((IOPIN0 & PIN_BUTTON) == 0) {
		return TRUE;
	}
	else {
		return FALSE;
	}
}
	\end{lstlisting}
This function check wether the button on the board is pressed.

    \subsection{RTC}
	\begin{lstlisting}[caption = {RTC initialization}]
	RTC_Time setTime, alarmTime, currentTime;


    CCR = 0x01;
	\end{lstlisting}

	First, the structs for time to set, alarm time and current time are created. The RTC interrupts are disabled by writing 0x0 to Interrupt Location Register (ILR).  The clocks tick counter is also reset by writing 0x02 to the Clock Control Register (CCR), and disabled by writing 0x0. Next, the Counter Increment Interrupt (CIIR), which is responsible for incrementing seconds, is disabled. 

Next, the prescaler is configured. Prescaler is a circuit used to reduce a high frequency electrical signal (for example from RTC) to a lower frequency by integer division. After calculating the value of a prescaler integer:
\[ (PCLK/32768) – 1 \]
and value of prescaler fraction:
\[ PCLK – ((PREINT + 1) * 32768) \]
Those values are written to Prescaler Integer Register (PREINT) and Prescaler Fraction Register (PREFRAC) registers.

Finally, the time is read from EEPROM, written into setTime struct and set using RTC\_SetTime function. At the end of initialization 0x01 is writtent to Clock Control Register (CCR), which enables the timer counters.

	\begin{lstlisting}[caption = {RTC struct}]
typedef struct
{
	uint8 seconds;
	uint8 minutes;
	uint8 hours;
} 
	\end{lstlisting}
	The struct consists of three 8 bit unsigned integer numbers, each one for seconds, minutes and hours fields.

	\begin{lstlisting}[caption = {Setting time in the register function}]
void RTC_SetTime(RTC_Time Time)
{
	SEC = Time.seconds;
	MIN = Time.minutes;
	HOUR = Time.hours;
}
	\end{lstlisting}
	The function uses data stored in the parsed struct and writes it to counters from Time Register Group. SEC and MIN are 6 bit, and the HOUR counter is 5 bit.

	\begin{lstlisting}[caption = {Setting alarm time in the register function}]
void RTC_Set_AlarmTime(RTC_Time AlarmTime)
{
	ALSEC = AlarmTime.seconds;
	ALMIN = AlarmTime.minutes;
	ALHOUR = AlarmTime.hours;
}
	\end{lstlisting}
	The function uses data stored in the parsed struct and writes it to counters from Alarm Register Group. Similarly to the Time Register Group, ALSEC and ALMIN are 6 bit, and the ALHOUR counter is 5 bit.

	\begin{lstlisting}[caption = {Getting time from the register function}]
RTC_Time RTC_GetTime(void)
{
	RTC_Time time;

	time.seconds = SEC;
	time.minutes = MIN;
	time.hours = HOUR;

	return time;
}
	\end{lstlisting}
	The function takes data from SEC, MIN and HOUR registers stored in Time Register Group, assigns it to the newly created RTC\_Time struct and returns it.

	\begin{lstlisting}[caption = {Getting alarm time from the register function}]
RTC_Time RTC_GetAlarmTime(void)
{
	RTC_Time AlarmTime;

	AlarmTime.seconds = ALSEC;
	AlarmTime.minutes = ALMIN;
	AlarmTime.hours = ALHOUR;

	return AlarmTime;
}
	\end{lstlisting}
	Analogously, to the RTC\_GetTime function, this function takes data from ALSEC, ALMIN and ALHOUR registers stored in Alarm Register Group, assigns it to the newly created RTC\_Time struct and returns it.

	\begin{lstlisting}[caption = {RTC\_Time struct comparator function}]
bool RTC_Compare(RTC_Time t1, RTC_Time t2)
{
	return (t1.seconds == t2.seconds
		&& t1.minutes == t2.minutes
		&& t1.hours == t2.hours);
}
	\end{lstlisting}
	The function is used to compare two RTC\_Time structs by comparing each field of the struct (seconds, minutes and hours).

    \subsection{Timer}

    For producing precise delay we use timer 0 of the microcontroller.
    
    To start, we reset and disable the timer, by writing $2$ (bit 1) to the timer control register (TCR). 
    Then, we can write the values controlling the delay.
    Firstly we write the prescaler to prescale counter (PC).
    Then, we write actual delay length to the match register 0 (MR0).
    We calculate the necessary number of ticks by multiplying the clock frequency by 5, which is the PLL multiplier, and then dividing by 1000 to get ticks per 1 ms instead of 1 s.

    Then, we write 6 (bits 1 and 2) to the match control register (MCR) to indicate that we want the timer to stop and reset at MR0 match.
    Finally we write 1 to the control register to enable the timer and start counting \cite[p.251]{lpc2148-manual}.

    \begin{program}[H]
        \code{11}{19}{timer.h}
        \caption{Timer delay function}
    \end{program}

    \section{Failure Mode and Effect Analysis}

    \subsection{Failure severity}

    The project depends on its components to be able to function properly.
    However, not all of them are equally impornant --- some are necessary for the usage of the project,
    while others just provide additional functionalities.

    The table bellow presents a list of components considered in failure effect analysis, as well as the significance of it's failure.

    \begin{table}[H]\centering
        \newcommand{\critical}{\color{red} Critical}
        \begin{tabular}{|l|c|}
            \hline
            \bf Component   &   \bf Severity                \\\hline
            Microcontroller &   \critical                   \\\hline
            Power Supply    &   \critical\footnotemark      \\\hline
            RTC             &   \critical                   \\\hline
            LCD Display     &   High                        \\\hline
            Buzzer          &   Medium                      \\\hline
            Button          &   Medium                      \\\hline
            Joystick        &   Medium                      \\\hline
            EEPROM          &   Low                         \\\hline
        \end{tabular}
        \caption{Severity of component's failure}
    \end{table}
    \footnotetext[1]{Long-term power supply failures are of critical severity, but in case of short pause in power delivery, the system is able to recover using the RTC and the data stored in EEPROM}

    The most crucial elements of the project are the microcontroller, power supply, and RTC.
    The microcontroller must not fail, because it is responsible for all of the logic of the device.
    The power supply must also be functioning, because there is no backup battery able to sustain the device operation.
    The RTC is necessary for the operation of the project, because it ensures the correctness of the told time.

    The failure of the LCD display is of high severity, because is is the main way of interacting with the clock.
    However, it is not of critical severity, because even without the display, if the alarm was set, it will ring at the specified time.

    The buzzer, button and joystick are used for the alarm, which is the additional functionality of the clock.
    Without them it is still possible to use the device for telling the current time.

    The EEPROM failure is of the low severity, because it is only used for backup storage in case of short power supply failure.
    In the normal circumstances, it's failure would not at all impact the usability of the device.

    \subsection{Failure detection}

    The failure of the microcontroller would be very easily noticeable, as it would probably result in a complete failure of the device.
    The lack of power 

    \addcontentsline{toc}{section}{References}
    \nocite{*}
    \bibliographystyle{unsrt}
    \bibliography{references}
\end{document}
